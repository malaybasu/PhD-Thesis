\chapter{Synopsis}
\textsc{Every living organism has} to adapt to its environment to
survive and flourish. Microbes show excellent adaptation
capabilities, brought about mainly, by their small genome size and
rapid cycles of cell divisions. Their gamut of adaptive abilities
is reflected in hyperthermophiles, which grow at temperatures as
high as 113\dg{} to psychrophiles, which grow at temperatures near
0\dg{} or below. To survive and flourish in these extreme
environments, they resort to various biochemical and genetic
tricks, much of which is still elusive. Thorough study of the
mechanisms by which microbes adapt to these harsh conditions is
important, not only to understand to what extent biology can
extend its limits in spite of being governed by unbreakable
physical rules, but also to answer the long standing
question---how and in what conditions life originated on this
earth?

The mechanism of \emph{cold-adaptation} is a complex phenomenon,
encompassing every aspect of the cellular biochemistry. If the
flexibility of the system determines the adaptability then the
most adaptive of all the known organism are the
\emph{psychrotrophs}, which grow at any temperatures from
0--30\dg{}\@. This dissertation is just a part of an ongoing
effort to understand the biology of a psychrotrophic bacterium,
\emph{Pseudomonas syringae} Lz4W, originally isolated from
Antarctic soil samples, with an aim to shed some lights on the
mechanism of cold-adaptation.

\subsection*{Objectives}

Transcription, carried out by the enzyme, DNA-dependent RNA
polymerase, is an essential biochemical process for an organism.
Earlier studies from our laboratory have demonstrated that RNA
polymerase from the Antarctic \bact{Ps} Lz4W has the ability to
transcribe at 0\dg{}, both \emph{in vivo} and \emph{in vitro}. The
broad objective of the present work was to understand the
mechanism of transcription at low temperature. The specific focus
of work presented in thesis was to assess the role of one of the
subunits of RNA polymerase, the \s{}, in the adaptive process of
\emph{P. syringae} Lz4W at low temperature.

$\sigmaup$ factor, a subunit of RNA polymerase, confers the
promoter recognition capabilities to the holoenzyme and is thought
to play a role in promoter melting. The melting of two strands of
DNA is the prime energy barrier that the organism has to overcome
to initiate transcription at low temperature. Moreover, a
particular $\sigmaup$ factor, \sigs{}, encoded by gene
\emph{rpoS}, is the master regulator of the stress-response in
Gram-negative bacteria. The majority of the transcription during
exponential growth in Gram-negative bacteria is carried out by the
principal $\sigmaup$ factor, coded by gene \emph{rpoD}, which is
replaced by \sigs{} during cessation of the growth, and under
variety of the stress conditions.

The major objectives of this thesis, therefore, were:

\begin{enumerate}

\item To clone the \emph{rpoD} and \emph{rpoS} genes from the
Antarctic psychrotroph \bact{Ps} Lz4W.

\item To compare these two $\sigmaup$ factors with their
mesophilic counterpart to elucidate the changes, if any, in these
$\sigmaup$ factors which might explain the enzymatic activity of
the RNA polymerase isolated from this organism.

\item To investigate the role, if any,  of the master regulator of
stress response, \sigs{} in the cold-adaptation of this organism.

\end{enumerate}


\subsection*{Structure of this dissertation}

\begin{description}

\item[Chapter 1] gives a general introduction to the field. The
intention has been not to be exhaustive, but to highlight the
salient observations to put the information in the later Chapters
in context. Wherever possible, references have been made to
reviews available in the field, and information supplementing the
reviews has been discussed.

\item[Chapter 2] discusses the materials used in this work and
experimental methods in details. Methods, more specific, such as
sequence analysis, have been discussed along with the results in
the corresponding Chapters.

\item[Chapter 3] describes the strategies and the procedures of
cloning the \emph{rpoD} and \emph{rpoS} homologs of \bact{Ps}
Lz4W. The genes encoding several other RNA polymerase subunits
from the bacterium have also been analyzed on Southern
hybridization to examine the possibility of cloning these subunits
using heterologous probes. Identification, obvious features of the
cloned genes, and database search results are also mentioned.

\item[Chapter 4] describes the detailed sequence analysis of the
cloned genes. Primary structures of RpoD and RpoS are compared
with the known homologs. Secondary structure of \emph{rpoS} mRNA,
which is known to play a crucial role in its regulation is
compared with its mesophile counterpart. A special discussion on
the phylogeny of \emph{rpoS} is also included.

\item[Chapter 5] describes the results of the effect of
\emph{rpoS} gene from \bact{Ps} Lz4W on several known promoters of
\emph{rpoS}-controlled \bact{Ec} genes. A possibility of promoter
discrimination within \emph{rpoS} regulon is also discussed.

\item[Chapter 6] describes the results of detailed phenotypic and
molecular characterization of \emph{rpoS} gene of \bact{Ps}
Lz4W\@. The level of the \sigs{} expression has been compared
between cells growing at 4\dg{} and at 22\dg{}\@. The \emph{rpoS}
gene of Lz4W has been disrupted, and the resulting mutant is
characterized for its ability to grow under various stress
conditions. The results of a detailed characterization of the
catalase enzyme of Lz4W is also discussed. Finally, the effect of
a chimeric \emph{rpoS}, created by fusion of N-terminal half of
Lz4W \emph{rpoS} and C-terminal half from \bact{Pa} \emph{rpoS},
on the growth properties of Lz4W has been discussed.

\item[Chapter 7] is a final overview of the results and the
conclusions. The implications of the findings in this work is
discussed in general and also with an evolutionary perspective; a
possible role of \s{} factors in cold adaption is proposed.


\end{description}

\subsection*{Summary of the results}

A fragment of \emph{rpoD} and \emph{rpoS} were amplified using
degenerate primers designed from the most conserved regions of
$\sigmaup$ factors. Using this amplified fragment and heterologous
gene as probes, a fragment of the \emph{rpoD} gene and the
full-length \emph{rpoS} gene were cloned by screening the genomic
library of Lz4W\@. Both these genes are highly similar to their
mesophilic counterpart. The cloned fragment of \emph{rpoD}
contained the major functionally important regions of the gene.
Sequence analysis did not reveal any special changes in the
primary sequence of the protein that might give some insight into
the ability of the Lz4W RNA polymerase to transcribe at low
temperature.

The \emph{rpoS}, interestingly, harbors an amber termination codon
at amino acid position 253 of the \emph{rpoS} reading-frame (334
aa), which putatively will produce a truncated \sigs{}, missing
region 4. The region is known to be involved in recognition of
$-$35 element of the promoter. Mutation of \emph{rpoS} gene has
been reported to arise during prolonged growth in stationary
phase. Such mutations confer a \emph{G}rowth \emph{A}dvantage at
\emph{S}tationary \emph{P}hase (GASP) phenotype to the bearer,
enabling it to outcompete the parental wild-type cells. Such
mutations have been described in several natural and laboratory
strains of \bact{Ec} and \emph{Salmonella}.

Remarkably, this mutated form of \bact{Ps} \emph{rpoS} showed the
ability to transcribe at least two \emph{rpoS}-controlled
promoters in \bact{Ec}, indicating that this \emph{rpoS} allele is
indeed functional. An anti-RpoS polyclonal antiserum generated
against \bact{Pp} \sigs{} detected a \U{31}{kDa} band in \bact{Ps}
Lz4W cell-extract, thus providing evidence that the amber in the
\emph{rpoS} reading-frame in Lz4W is not suppressed but results in
the production of a truncated \sigs{}\@. A \emph{rpoS}-null mutant
of Lz4W was generated by disrupting the \emph{rpoS} gene before
the amber using homologous recombination. When compared with the
wild-type Lz4W, the null-mutant was only marginally sensitive to
low temperature but severely defective in its ability to grow in
low pH both at 22\dg{} and at 4\dg{}\@. This results, therefore,
reconfirms that the truncated \emph{rpoS} is indeed functional in
\bact{Ps} Lz4W.

Catalases are known to be controlled by \sigs{} in \bact{Ec}. In a
study of \bact{Ps} Lz4W catalase, it was found that the bacteria
possess only one isoform of catalase detectable by activity
staining in the native gel. Although, the expression of this
catalase was growth phase dependent, it is most probably not
controlled by \emph{rpoS}\@. In \emph{rpoS} null-mutant, the
activity of this catalase was consistently higher than the
activity in wild-type cells.

When \sigs{} protein levels were compared between cells growing at
22\dg{} and at 4\dg{}, it was found that \sigs{} expression is
actually suppressed during growth at low temperature. This result
is in contrary to finding in \bact{Ec}, where \sigs{} level was
found to increase during growth a low temperature. To check the
effect, what the full-length \emph{rpoS} might have on the ability
of \bact{Ps} Lz4W to grow at low temperature, a chimeric
\emph{rpoS} was created by fusing the N-terminal half of Lz4W
\emph{rpoS} with C-terminal half from \bact{Pa} \emph{rpoS}\@.
Such a chimeric \emph{rpoS} would in all aspect be similar to the
native \emph{rpoS} of Lz4W, except the amber in its reading-frame.
This chimeric \emph{rpoS} was found to produce a full-length
\sigs{} in Lz4W.

When transformed into the \emph{rpoS}-null mutant, the chimeric
\emph{rpoS}, instead of relieving the mild cold-sensitive
phenotype of the mutant, actually exacerbated the phenotype. The
amber-mutated native \emph{rpoS} of Lz4W, when supplied in
\emph{trans}, also aggravated the cold-sensitivity of the mutant,
however, to a much lesser extent than the full-length chimeric
\emph{rpoS} supplied in \emph{trans}. When compared with the level
of \sigs{} expression from the genomic copy of the \emph{rpoS},
\sigs{} protein level produced from plasmid was found to be five
folds higher in case of native, amber-mutated \emph{rpoS} of Lz4W,
and twenty folds higher in case of chimeric, full-length
\emph{rpoS}.

\subsection*{Conclusions}

The major conclusions from this study are:

\begin{enumerate}

\item Both \e{rpoD} and \e{rpoS} from \e{P. syringae} Lz4W are
very similar to their meso\-philic counterpart.

\item The \e{rpoS} gene of \e{P. syringae} Lz4W harbors an amber
mutation in its reading-frame.

\item The \emph{rpoD} is not the determinant of the ability of RNA
polymerase from \bact{Ps} Lz4W to transcribe at low temperature.
At least, the major functional regions are not different from
their mesophilic counterpart. The cold-transcription ability,
therefore, lies in some other subunit(s) of RNA polymerase, or in
some extraneous factor(s).

\item \sigs{} level is tightly regulated and suppressed in Lz4W
during growth at low temperature. An amber mutation might be a
gene regulatory mechanism to control the level of \sigs{}, or
enabling promoter discrimination during growth at low temperature.
Both high level of expression and the absence of \sigs{}, is
detrimental to growth of \bact{Ps} Lz4W at low temperature and in
other stressful conditions, such as, low pH\@. An amber-mutated
\sigs{} is, therefore, another fascinating way of by which
bacteria adapt to cold. \ding{45}

\end{enumerate}
