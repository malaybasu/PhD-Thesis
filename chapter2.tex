\chapter{Materials and Methods}
%\minitoc

\section{Chemicals}

Agarose, ethidium bromide, X-gal, IPTG, SDS, BSA, acrylamide,
bis-acry\-la\-mide, $\betaup$-mercapto\-ethanol, ammonium
persulphate, TEMED, Tris, EDTA, and MOPS were procured from
various vendors including Gibco-BRL, Sigma, Calbiochem and others.
Tryptone, yeast extract, peptone and agar powder for preparation
of culture media were obtained from Hi Media (India). All
restriction enzymes, Klenow polymerase, T4 DNA ligase, and
polynucleotide kinase were purchased from New England Biolabs Inc.
(USA). Lysozyme, DNase I, RNase, Proteinase K were purchased from
Roche (Germany). \textit{Taq} DNA polymerase, DNA and protein
molecular weight markers were purchased from Bangalore Genei
(India). DNA and RNA purifying kits were purchased from QIAGEN and
Promega. Phage packaging kit was from Stratagene (USA). The DNA
random primer kit for labelling DNA, and radiolabelled nucleotides
were from BRIT (India). Sephadex G-50 and G-25 were from Pharmacia
Bio\-tech (Sweden). Custom made oligonucleotides were obtained
either from Oswel (UK) or from in-house oligo synthesis facility.
Nitrocellulose, Hybond{\scriptsize\texttrademark}-N\su{+} and
PVDF{\scriptsize\texttrademark} membranes were from Amersham (UK)
and Whatman filter paper from Whatman International ltd. (USA).
X-ray films were obtained from Konica Corporation (Japan). All
other chemicals were purchased from local manufacturers and were
of analytical grade.

\section{Software} All sequence analyses were performed over WWW
using SeWeR~\citep{Basu2001} interface. In some special cases,
custom-made Perl scripts were used. Other specific softwares are
mentioned in this text as and when required.

\section{Oligonucleotides}

The oligonucleotides used in this study are listed in
Table~\ref{chap2_oligos}.

\begin{table}[tbp]
\linespread{1}\normalsize
\renewcommand{\arraystretch}{1.5}
\begin{minipage}[c]{\textwidth}
\renewcommand{\footnoterule}{}
\caption{Oligonucleotides used in this study} \label{chap2_oligos}
\begin{narrow}{-1in}{-1in}
\centering
\begin{small}
\begin{tabular}{@{}lp{4.5in}@{}}\toprule
\textbf{Name} & \textbf{Sequence\protect\footnote{From 5$'$ to
3$'$ direction.}}\\\midrule ``\emph{rpoD} box'' (Probe1) &
GCTTGGCGGATCCACCAGGTGGCGTAGGT\\
``\emph{rpoD} box'' (Probe2) &
GCTTGICIIATCCACCAIGTIGCITAIGT\protect\footnote{I stands for inosine.}\\
RPODFP1 &
AAGTTCTC(G/A/T/C)AC(C/G)TA(T/C)GC(A/T/G/C)ACC(G)T\-G\-G\-T\-G\-G\-A\-T\\
RPODFP2 & AAGTTCTC(C/G)ACCTACGC(A/C)ACCTGGTGGAT \\
RPODRP1 &
GCCTT(C/G)GCTTC(G/A)AT(C/T)TG(G/C/A/T)CG(G/A)\-AT\-(A/C/G/T)\-C\-(G/T)(C/T)TC\\
RPODRP2 &
GCCTT(C/G)GC(C/T)TCGATCTG(A/G)CGGAT(A/C)CG(C/T)TC\\
RPOSKOF &
CCG\textbf{\underline{GAATTC}}CTACTACTGCCCGCACCAAGT\protect\footnote{\emph{Eco}RI
site indicated in boldface and underlined.} \\
RPOSKOR &
CCC\textbf{\underline{AAGCTT}}AAAAGCATGCGCTTGACCTC\protect\footnote{\emph{Hin}dIII
site indicated in boldface and underlined.}\\
\bottomrule
\end{tabular}
\end{small}
\end{narrow}
\end{minipage}
\linespread{1.1}\normalsize
\renewcommand{\arraystretch}{1.0}
\end{table}
\section{Plasmids and cosmid}

\begin{description}

    \item [\textbf{pBluescriptII KS+}] (Stratagene) is a high-copy-number, ColE1-based
    vec\-tor, which confers ampicillin resistance (Amp\su{r}) to the host cell.
It also carries a
        multiple cloning site region in \textit{lacZ$\alphaup$}
        fragment enabling blue-white screening of the recombinant
        clone.
    \item[\textbf{pUC19}] (Pharmacia) is a high-copy-number, ColE1-based vector containing Amp\su{r} marker, and carries
    \textit{lacZ}$\alphaup$ fragment enabling blue-white screening
    of the recombinant clone.

    \item [\textbf{pDB18R}] \citep{Fujita1994} is a high-copy-number plasmid
        construct derived from pTZ18R (Pharmacia). It carries a
        \unit[1.8]{kb} \textit{Kpn}I-\textit{Hind}III fragment containing
        \bact{Pa} \textit{rpoS} promoter and
        structural gene.

    \item [\textbf{pASB3}] \citep{Tanaka1991} is plasmid pTZ18R carrying a
        \unit[2.3]{kb} \textit{Pst}I DNA fragment containing \textit{rpoD} of
        \bact{Pa}.

    \item [\textbf{pLAFR3}] \citep{Staskawicz1987} is a broad-host-range cosmid with Tc\su{r} marker; allows blue-white screening because of the presence of
\emph{lacZ}$\alphaup$ fragment.

    \item [\textbf{pGEMD}] \citep{Igarashi1991} is derived from
    pGEMEX1 (Promega) containing a \U{2.1}{kb} \emph{Hin}dIII
    fragment spanning the full-length \emph{rpoD} of \bact{Ec}.

    \item[\textbf{pGEMAX185}] \citep{Igarashi1991} is derived from
    pGEMEX1 (Pro\-mega) containing \U{1.2}{kb} \emph{Xba}I
    fragment spanning the full-length \emph{rpoA}
    of \emph{E. coli}.

    \item[\textbf{pGEMBC}] \citep{Igarashi1991} is derived from
    pGEMEX1 (Promega) containing a \U{10.2}{kb} \emph{Hin}dIII
    fragment spanning the full-length \emph{rpoB}, and \emph{rpoC}
    gene of \emph{E. coli} with upstream \emph{rplL}.

    \item[\textbf{PGEM-T}] (Promega) is a TA cloning vector of size \U{3003}{bp}, and is used to clone
    PCR product with A-overhang. It contains Amp\su{r} marker and
    \textit{lacZ}$\alphaup$ for screening of recombinant clones.

    \item[\textbf{pCL1920}] \citep{Lerner1990} is a low-copy-number
    vector with pSC101 replicon ($\sim$5 copies/cell). It
    carries streptomycin/spectinomycin resistance mar\-ker (encoded
    by \emph{aadA}), and also carries \emph{lacZ}$\alphaup$ that
    allows blue-white screening of the recombinants.

    \item[\textbf{p4A4}] is pBluescriptII KS$+$ containing a
    \emph{Sal}I--\emph{Pst}I insert of \U{$\sim$2}{kb} containing C-terminal half of \emph{rpoD} homolog of \emph{P.
    syringae} (Lz4W).

    \item[\textbf{p4C12}] is pBluescriptII KS+ containing a
    \U{$\sim$2.1}{kb} \emph{Pst}I fragment with the
    full-length \emph{rpoS} of \emph{P. syringae} (Lz4W).

    \item[\textbf{pRPOH5}] \citep{Aramaki1996} is plasmid pTZ18R containing a
    \U{1.9}{kb} \emph{Sal}I--\emph{Pst}I fragment spanning the
    full-length \emph{rpoH} of \bact{Pa}.

    \item[\textbf{pGL10}] \citep{Bidle1999} is a broad-host-range
    cloning vector with IncP replicon. It has a genotype,
    \emph{tra}\su{$-$} \emph{mob}\su{$+$} and Km\su{r}.

    \item[\textbf{pGLSIGS}] is pGL10 containing the insert of
    p4C12.

    \item[\textbf{pXRPOS}] is pGL10 containing
    \emph{Pst}I--\emph{Bam}HI fragment from p4C12 and
    \emph{Bam}HI--\emph{Hin}dIII fragment from pDB18R, ligated in
    tandem, which results in a chimeric \emph{rpoS}.

    \item[\textbf{pME3088}] \citep{Schnider2000} is a suicide vector with ColE1 replicon with genotype,
    \emph{RK2-Mob} Tc\su{r}.

    \item[\textbf{pMERPOS$'$}] is pME3088 containing a
    \U{$\sim$750}{bp} internal fragment of \emph{rpoS} gene of
    Lz4W.


\end{description}




\section{Culture media}

All media were prepared using water purified through MilliQ water
purification system (Millipore).

\begin{longtable}{lll}\addlinespace
\multicolumn{3}{@{}l}{\textbf{1. Culture media for \bact{Ps}
(Lz4W)}}\\\addlinespace
\multicolumn{3}{l}{\textbf{Antractic Bacteria Medium (ABM) broth:}}\\
    &   Peptone &  \unit[5]{g} \\
    & Yeast extract & \unit[2.5]{g} \\
       & Water to   &    \unit[1000]{ml}\\
       &     \multicolumn{2}{l}{pH adjusted to 7.0-7.2 with \unit[1]{M} NaOH.}\\
       &     \multicolumn{2}{l}{Sterilized by autoclaving.}\\
       \addlinespace

{\bfseries ABM agar:} & ABM  &  \ml{100}\\
              &  Agar & \U{1.5}{g}\\
              \addlinespace\midrule\addlinespace

\multicolumn{3}{@{}l}{\textbf{2. Culture media for
\bact{Ec}}}\\\addlinespace
{\bfseries LB medium:} &  Tryptone & \unit[10]{g} \\
                        & Yeast extract & \unit[5]{g}\\
                       & NaCl  &  \unit[10]{g}\\
                & Water to & \unit[1000]{ml}\\
    &     \multicolumn{2}{l}{pH adjusted to 7.0-7.2 with \unit[1]{M} NaOH.}\\
       &     \multicolumn{2}{l}{Sterilized by
       autoclaving.}\\\addlinespace

    {\bfseries LB agar:} &   \multicolumn{2}{l}{LB medium with 1.5\%
    (\nicefrac{w}{v}) agar.}\\\addlinespace


    {\bfseries LB soft agar:} & \multicolumn{2}{l}{LB medium with 0.6\% (\nicefrac{w}{v}) agar.}\\\addlinespace

    \textbf{SOB medium:} & Tryptone  & \U{20}{g} \\
                        & Yeast extract & \U{5}{g}\\
                        & NaCl          & \U{0.5}{g}\\
                        & Water to      &\ml{1000} \\
                        & \multicolumn{2}{p{3in}}{pH adjusted to 7. Sterilized by autoclaving, and just before use, \ml{5} of sterile \U{2}{M} MgCl\sub{2} was added.}\\\addlinespace

    \textbf{SOC medium:} & SOB medium  & \ml{1000} \\
                        & Glucose (\U{1}{M}) & \ml{20}
                        \\\addlinespace

    {\bfseries Z broth:} & \multicolumn{2}{l}{LB medium supplemented with 0.5\% (\nicefrac{v}{v}) CaCl\sub{2}
    (\U{0.5}{M}).}\\\addlinespace

    {\bfseries Z agar:} & \multicolumn{2}{l}{Z broth with 0.75\%
    (\nicefrac{w}{v}) agar.}
    \addtocounter{table}{-1}
\end{longtable}

\section{Buffers and solutions}
    \label{buffers}
    \begin{longtable}{llll}
   \multicolumn{4}{@{}l}{\textbf{1. Plasmid isolation solution for alkaline
    lysis}}\\\addlinespace
  &\textbf{Solution 1:} &  Glucose & \mM{50}\\
  &                     &  Tris-Cl (pH 8.0) & \mM{25}\\
   &                & EDTA (pH 8.0) & \mM{10}\\\addlinespace
   & \textbf{Solution 2:} & NaOH &   \U{0.2}{M} \\
    &                   &  SDS & 1\% \\\addlinespace
    &\textbf{Solution 3:} & Potassium acetate (\U{5}{M}) & \ml{60}\\
    &                 & Glacial acetic acid & \ml{11.5}\\
     &                & Water & \ml{28.5}\\
     &                & \multicolumn{2}{p{3in}}{The pH of the solution is approximately
                     4.8.}\\\addlinespace\midrule\addlinespace

   \multicolumn{4}{@{}l}{\textbf{2. Electrophoretic buffer for nucleic acids}}\\\addlinespace
    &  \textbf{TAE:} & Tris-acetate & \unit[40]{mM} \\
    &               &  EDTA        & \unit[2]{mM}  \\
    &               & \multicolumn{2}{p{3in}}{Prepared as 50 $\times$
        concentrated stock solution and used as 0.5 $\times$
        concentration.}\\\addlinespace

    &  \textbf{TBE:} & Tris-borate  & \unit[90]{mM}\\
    &               &  EDTA        &  \unit[2]{mM} \\
    &               &\multicolumn{2}{p{3in}}{Prepared as 10 $\times$ stock solution and used as 0.5--1 $\times$
concentration.}\\\addlinespace\midrule\addlinespace

\multicolumn{4}{@{}l}{\bfseries 3. Hybridization solution}\\
    & &Na$_{2}$HPO$_{4}$ & \unit[0.5]{M}\\
    & & SDS              & 7\%\\\addlinespace\midrule\addlinespace

 \multicolumn{4}{@{}l}{\bfseries 4. Buffers for transformation
 }\\\addlinespace
 & {\bfseries TB:}& PIPES (pH 6.7) & \mM{10} \\
 & &  CaCl$_{2}\cdotp$2H$_{2}$O & \mM{15}\\
 & &   KCl  &    \mM{250}\\
 & &MnCl$_{2}$& \mM{55}\\\addlinespace

   & \textbf{RF1:} & RbCl & \unit[100]{mM} \\
   &               &MnCl$_{2}\cdotp$4H$_{2}$O &\unit[50]{mM}\\
    &               & potassium acetate &    \unit[30]{mM}\\
     &              & CaCl$_{2}\cdotp$\-2H$_{2}$O &   \unit[10]{mM}\\
     &             & Glycerol  & 15\% \nicefrac{v}{v}\\
   &              &\multicolumn{2}{p{3in}}{pH adjusted to 5.8 by glacial acetic acid, filter-sterilized and stored frozen at
-20\,$^\circ$C.}\\\addlinespace


  &\textbf{RF2:} &  MOPS & \unit[10]{mM}\\
  &             & CaCl$_{2}\cdotp$2H$_{2}$O & \unit[75]{mM}\\
   &            & RbCl         &  \unit[10]{mM} \\
   &              & Glycerol   & 15\% \nicefrac{v}{v}\\
   &              &\multicolumn{2}{p{3in}}{pH adjusted to 6.5 by adding \unit[1]{M} KOH, filter-sterilized and stored frozen at
-20\,$^\circ$C.}\\\addlinespace\midrule\addlinespace

 \multicolumn{4}{@{}l}{\textbf{5. Native polyacrylamide gel (10\%)}}\\
 & & \multicolumn{2}{l}{For \ml{10}---}\\
& & 30\% acrylamide mix & \ml{3.3}\\
& & \U{1.5}{M} Tris (pH 8.8) &\ml{2.5}\\
& & APS (10\%)               & \mul{100} \\
& &TEMED                    &\mul{4} \\
& & water                  &\ml{4.1}\\
& &\multicolumn{2}{p{3in}}{The
        gel was run in buffer containing \mM{25} Tris (pH 8.3), \mM{250}
        glycine.}\\\addlinespace\midrule\addlinespace

\multicolumn{4}{@{}l}{\bfseries 6. Reagents for
SDS-PAGE}\\\addlinespace
 \multicolumn{2}{l}{\textbf{Resolving gel (10\%):}}  & \multicolumn{2}{l}{For \ml{10}---}\\
& & water                  &\ml{4}\\
& & 30\% acrylamide mix & \ml{3.3}\\
& & \U{1.5}{M} Tris (pH 8.8) &\ml{2.5}\\
& &SDS (10\%)                      &\mul{100}\\
& & APS (10\%)               & \mul{100} \\
& &TEMED                    &\mul{4} \\\addlinespace

 \multicolumn{2}{l}{\textbf{Stacking gel (5\%):}} & \multicolumn{2}{l}{For \ml{10}---}\\
& & water                  &\ml{6.8}\\
& & 30\% acrylamide mix & \ml{1.7}\\
& & \U{1.0}{M} Tris (pH 6.8) &\ml{1.25}\\
& &SDS (10\%)                      &\mul{100}\\
& & APS (10\%)               & \mul{100} \\
& &TEMED                    &\mul{10} \\\addlinespace

 \multicolumn{2}{l}{\textbf{Running buffer:}} & Tris (pH 8.3) & \mM{25} \\
 & &  Glycine & \mM{250} \\
 & & SDS & 0.1\%\\\addlinespace

 \multicolumn{2}{l}{\textbf{1 $\times$ Sample buffer:}} & Tris (pH 6.8)& \mM{50}\\
      & & Dithiothreitol  &         \mM{100} \\
& &      SDS       &       2\% \\
 & & Bromophenol blue & 0.1\% \\
    & & Glycerol  &10\%\\\addlinespace\midrule\addlinespace

 \multicolumn{4}{@{}l}{\textbf{7. Buffers and solutions for immunoblotting}}\\\addlinespace

 \multicolumn{4}{@{}l}{\bfseries Transfer buffer for semi-dry transfer of
 proteins:}\\
    & &  Glycine &  \unit[39]{mM}\\
    & &   Tris   &   \unit[48]{mM}\\
     & &  SDS    &  0.037\% \\
     & & Methanol & 20\%\\\addlinespace


    &\textbf{TBS:}   & Tris-HCl (pH 7.5) & \unit[10]{mM} \\
    &               & NaCl              & \unit[150]{mM}\\
    &               & \multicolumn{2}{l}{Made as 10 $\times$
    stock.}\\\addlinespace

    &\textbf{TBS-T:} &\multicolumn{2}{l}{TBS with 0.1\% Tween-20~\protect\footnote{Tween is a registered trademark of ICI Americas
        Inc.}}\\\addlinespace

\multicolumn{4}{@{}l}{\bfseries Alkaline phosphatase(AP) buffer:}\\
& & Tris-Cl (pH 9.5)& \unit[100]{mM}\\
 & &  NaCl           &  \unit[100]{mM}\\
   & &      MgCl$_{2}$. &
   \unit[5]{mM}\\\addlinespace\midrule\addlinespace

 \multicolumn{4}{@{}l}{\textbf{8. Buffers for phage handling}}\\\addlinespace
 & \textbf{SM buffer:} & For \ml{1000} buffer: & \\
 & &NaCl & \U{5.5}{g} \\
 & & MgSO$_{4}\cdot$7H$_{2}$O & \U{2.0}{g}\\
 & & \U{1}{M} Tris-HCl (pH 7.5)& \ml{50}\\
 & & Gelatin (2\% \nicefrac{w}{v}) & \ml{50}\\
& &\multicolumn{2}{p{3in}}{Water to 1 liter. Strerilized by
autoclaving.} \\\addlinespace

 & \textbf{Citrate buffer:} & Citric acid (\U{0.1}{M}) & 4.7 volumes \\
& & Sodium citrate (\U{0.1}{M}) & 15.4
volumes\\\addlinespace\midrule\addlinespace

   \multicolumn{4}{@{}l}{\textbf{9. Other buffers}}\\\addlinespace


    &\textbf{TE:} & Tris-Cl (pH 8.0)& \unit[10]{mM} \\
    &            &    EDTA         & \unit[1]{mM}\\\addlinespace

    & \textbf{20 $\times$ SSC:} &  NaCl & \unit[173.5]{g}\\
    &                          &  Sodium citrate & \unit[88.2]{g} \\
     &                          & Water to       &  \unit[1000]{ml}\\
     &                          & \multicolumn{2}{l}{pH adjusted to 7.0 with
        NaOH.}\\\addlinespace
\addtocounter{table}{-1}
\end{longtable}



\section{Bacterial strains}

\textit{E. coli} and \textit{P. syringae} strains used in this
study are listed in Table \ref{ecolistrains}.
\begin{table}[tbp]
\begin{minipage}[c]{\linewidth}
\renewcommand{\footnoterule}{}
\caption{Bacterial strains used in this study}
\label{ecolistrains}
\begin{narrow}{-1in}{-1in}
\centering
\begin{small}
\linespread{1}\normalsize
\renewcommand{\arraystretch}{1.6}
\begin{tabularx}{6in}{@{}l>{\raggedright\arraybackslash}p{3.3in}>{\raggedright\arraybackslash}X@{}}\toprule
 \multicolumn{1}{c}{\textbf{Strains}}  & \multicolumn{1}{c}{\textbf{Genotype/Description}} & \multicolumn{1}{c}{\textbf{Source/Reference}} \\
 \midrule
\multicolumn{3}{@{}l}{\bfseries\textit{Escherichia coli}
strains\protect\footnote{All the strains are F$^{-}$ unless
stated.}} \\ DH5$\alphaup$ &\textit{$\Delta$(argF-lac)}U169
\textit{supE44 hsdR17 recA1 endA1 gyrA96 thi-1 relA1
$\phi$}80d\textit{lacZ$\Delta$}M15
&\multicolumn{1}{l}{\citet{Hanahan1983}}\\ DH10B\footnote{DH10B is
a trademark of Invitrogen corporation.} & \textit{$\Delta$(mrr
hsdRMS mcrBC) mcrA $\Delta$lacX74 deoR recA1 endA1 araD139
$\phi$}80d\textit{lacZ $\Delta$}M15 \textit{$\Delta$(ara, leu)7697
galU galK rpsL nupG} & \citet{Grant1990}
\\
MC4100         & \textit{$\Delta$(argF-lac)U169} \textit{rpsL150
relA1 araD139 flb5301 deoC1 ptsF25}  & \citet{Casadaban1976}
\\
RH90            & MC4100 \textit{rpoS359}::Tn\textit{10} &
\citet{Barth1995}\\ RH100\protect\footnote{The Tn\emph{10} allele
in RH100 was originally designated \emph{zfi-3251}::Tn\emph{10},
based on the calibration in an earlier edition of the \bact{Ec}
K-12 linkage map.}     & MC4100
$\Delta$(\emph{nlpD--rpoS})\emph{360 zgc-3251}::Tn\emph{10} &
\citet{Hengge1993}\\ GJ2733          & MC4100 \textit{csiD::lac
rpoS359}::Tn\textit{10} & \citet{Rajkumari2001} \hfill ~
\\
GJ2734          & MC4100 \textit{osmY::lac rpoS359}::Tn\textit{10}
&  \citet{Rajkumari2001}
\\
GJ2782  & RH100 \emph{ara}\su{$+$} [$\lambda$
\emph{katE}::\emph{lac}(Km)] & \citet{Rajkumari2002}\\
 JM101           & {\itshape supE44 thi
$\Delta$\textup{(}gpt-lac\textup{)}5 \textup{F$'$[}traD36
proAB$^{+}$ lacI$^{q}$ lacZ$\Delta$\textup{M15]}} &
\citet{Messing1979}\\
MJMRH           & JM101 \textit{rpoS}::Tn\textit{10} & This
    study.\\
    S17-1           & \textit{pro recA1 RP4-2 integrated (Tc::Mu) (Km::}Tn\textit{7)}
    (Sm\su{r}Tp\su{r})& \citet{Simon1983} \\
    ZK918 & W3110 $\Delta$\textit{lacU169 tna-2} $\lambda$MAV103
    \textit{rpoS::kan} & \citet{Bohannon1991} \\
    UQ285 & P90A5 \emph{lacZ4} Lam\su{-} \emph{rpoD285}(ts)
    \emph{argG75} & \citet{Isaksson1977}\\

    \multicolumn{3}{@{}l}{\bfseries\textit{Pseudomonas syringae}
    strains} \\
    Lz4W & Natural isolate from Antarctica & \citet{Shivaji1989}
    \\

    MBLz4W & Lz4W \textit{rpoS::}pME3088 (Tet\su{r}) & This study.
    \\
 \bottomrule
 \end{tabularx}
 \end{small}
\end{narrow}
\end{minipage}

\linespread{1.1}\normalsize \renewcommand{\arraystretch}{1.0}
\end{table}




\section{Antibiotics}

\begin{table}[tbp]
\caption{Antibiotics used for this study} \label{antibiotics}
\centering
\begin{small}
\linespread{1.0}{\normalsize}
\renewcommand{\arraystretch}{1.3}
\begin{tabularx}{\linewidth}{@{}XXXX@{}}\toprule
   & \textbf{\emph{E. coli} ($\muup$g/ml)}& \textbf{\emph{P. syringae} ($\muup$g/ml)} &
   \textbf{Solvent} \\ \midrule \addlinespace
    Ampicillin      & 100 & Naturally resistant & water \\
 %   Chloramphenicol & 20  & -                   & ethanol \\
 %   Gentamicin      & 15  &   15                & water  \\
    Kanamycin       & 50  &   50                & water  \\
  %  Rifampicin      & -   &  100                & methanol \\
    Spectinomycin   & 50  &   -                 & water  \\
    Tetracycline    & 20  &   20                & methanol \\
    \addlinespace
\bottomrule
 \end{tabularx}
\renewcommand{\arraystretch}{1.0}
\linespread{1.1}
\end{small}
\end{table}

Antibiotics used in this study and their concentrations are shown
in Table \ref{antibiotics}.

\section{Culturing of bacteria}
\label{chap2:culture} Typically, the growth measurement was
carried out by monitoring the turbidity of the culture, measured
as optical density (OD) at \U{600}{nm}, in rich medium (LB for
\bact{Ec}; ABM for \bact{Ps}). For growth comparison of different
strains, pre-inoculum of bacteria from the same stages of growth
was diluted 1000 times in \ml{100--150} medium, taken in \ml{500}
conical flask. The flasks were continuously shaken at \U{200}{rpm}
at the required temperature (22\dg{} or 4\dg{} for Lz4W). The
samples were withdrawn at regular interval and OD$_{600}$ was
measured either undiluted or by diluting the culture 5 times with
water. In the latter case, absorbance was calculated by
multiplying the OD obtained by the dilution factor. All the
measured samples were measured either diluted or undiluted
throughout the measurement. The undiluted culture OD$_{600}$
maximum was 1.8 for Lz4W, which was equivalent to OD$_{600}$ 3 for
diluted samples.

\section{Preparation of crude cell extract}

\label{crude} Cells were grown to required growth phase and
harvested. The cell pellet was then resuspended  in phosphate
buffer (\U{50}{mM} potassium phosphate, pH 7.0; \U{5}{mM} EDTA;
10\% glycerol; \muM{25} phenylmethylsulfonyl fluoride) to an
estimated OD of 10. Cells were then sonicated with Branson W-250
sonicator on ice. Typically cells were given four rounds of five
\U{2}{s} pulses with constant duty cycles and microtip settings of
2--3. Debris was pelleted by centrifugation at 4\dg{} for
\U{10}{minutes} at \g{12,000}. The supernatant was collected in a
separate tube and stored at -70\dg.

\section{Protein estimation}

The concentration of the protein in the crude cell extract were
determined by Bradford method \citep{Bradford1976} by BIO-RAD
dye-reagent{\scriptsize\texttrademark} according to the supplier's
protocol.

\section{Enzyme assays}

\subsection{$\betaup$-galactosidase assay}
Assays for determination of $\betaup$-galactosidase enzyme
activity in cultures were performed as described in
\citet{Miller1992} and the activity values are calculated in
\emph{Miller} units, as described therein.

\subsection{Catalase assay}

\subsubsection{Hydrogen peroxide bubbling test}

Quick scoring for catalase phenotype was done by placing a drop of
H$_{2}$O$_{2}$ (30\% \nicefrac{v}{v}) directly on the bacterial
colony as mentioned in \citet{Mulvey1988}. The rate of bubble
formation was taken as measure of catalase activity.

\subsubsection{Spectrophotometric assay}
\label{chap2:spec_catalase}

Total catalase activity measurement was carried out as described
in \citet{Visick1997}, which was based on \citet{Beers1951}.
Briefly, \mul{25} of the crude extract was diluted to \ml{1.5}
with \mM{50} potassium phosphate buffer, pH 7.0. Then \mul{2.5} of
30\% H$_{2}$O$_{2}$ was added and the absorbance of the samples at
\U{240}{nm} was measured every \U{15}{s} for \U{1}{min}. The
specific activity of the catalase ($\muup$mol of H$_{2}$O$_{2}$
decomposed/min/mg of total protein) was then calculated as
follows:

\begin{center}

$\frac{\textrm{1,000}\ \times\ \Delta
\textrm{A}_{\textrm\scriptsize 240}\ /\
\textrm{min}}{\textrm{43.6}\ \times\ \textrm{mg\ of\ protein} /
\textrm{ml\ of\ reaction\ mixture}}$

\end{center}

\section{Activity staining for catalase}
\label{chap2:activity_staining}

Negative staining of catalase activity on native gel was carried
out as described previously \citep{Gregory1974}. Although the same
group published an improved protocol later in \citet{Clare1984},
we found the older protocol gives consistent results.

In this method, the crude cell extract (Section \ref{crude}) was
separated on 8--10\% nondenaturing polyacrylamide gel (see
Section~\ref{buffers}) with no stacking gel. The glycerol in the
extraction buffer (Section \ref{crude}) was enough to load the
samples in the gel. A separate well was generally used to load a
mixture of bromophenol blue in 20\% glycerol to follow the
migration. The gel was run at constant current (\U{20}{mA}) for
required amount of time, at 4\dg{}\@. After the run, the gel was
soaked in diaminobenzidine (\U{0.5}{mg/ml}), horseradish
peroxidase (\U{50}{$\muup$g/ml}) in \mM{50} potassium phosphate
(pH 7.0) for \U{1} h at room temperature. The gel was then rinsed
and soaked in \mM{20} H$_{2}$O$_{2}$ in phosphate buffer until
staining was complete. The catalase bands appears as unstained
white bands on dark background.

\section{Genetic techniques}

\subsection{Bacterial conjugation}
\label{conjugation}

Plasmids were mobilized to Lz4W by biparental mating between Lz4W
and \bact{Ec} \mbox{S17-1}, transformed with the suitable plasmid.
The donor (S17-1, transformed with suitable plasmid) and the
recipient (Lz4W) strains were grown in \ml{3} of appropriate
medium supplemented with required antibiotics to mid-logarithmic
phase (OD$_{600}$ 0.6--0.9). The cells were pelleted by
centrifugation at \U{6,000}{rpm} for \U{5}{min} at 4\dg{}, and
were washed with \ml{1.5} of sterile ABM or LB broth. The cell
pellets were resuspended in \mul{100} of ABM or LB, and the donor
and the recipient were mixed in the ratio of 1:5
(\nicefrac{v}{v}). From this mixture, \mul{50}  was spotted onto
Hybond{\scriptsize{\texttrademark}} N\su{+} membrane (Amersham
Life Sciences, Buckinghamshire, UK) placed on ABM agar. Following
incubation for \U{24--72}{h} at 22\dg{}, the cells were scraped
off from the membrane with sterile toothpicks, resuspended in ABM,
and appropriate dilutions were plated on selection media. The
plates were incubated at 22\dg{} for \U{48}{h} to obtain
exconjugants.

\subsection{P1 lysate preparation on RH90}
\label{rh90_lysate}

From an of overnight culture of the strain RH90 (see Table
\ref{ecolistrains}) \ml{0.3} in Z broth was mixed with 10\su{7}
plaque forming units (pfu) of a stock P1 lysate prepared on strain
MG1655. Adsorption was allowed to occur at 37\dg{} for \U{20}{min}
and the lysate was prepared in the following way.

To \ml{0.3} of the infection mixture, \ml{10} of Z broth was added
and incubated at 37\dg{} with slow shaking until growth followed
by visible lysis of the culture occurred (\U{4--6}{h}). The lysate
was treated with \ml{0.3} of chloroform, centrifuged  and the
clear lysate was stored at 4\dg{} with chloroform.

To measure the titer of the P1 phage in the lysate, titration was
done using a P1-sensitive indicator strain such as MG1655.
Aliquots of \mul{100} each of serial dilutions (typically
10\su{5}--10\su{6}) were mixed with \mul{100} of the fresh culture
grown in Z broth. After \U{15}{min} adsorption at 37\dg{} without
shaking, each mixture was added into soft agar, overlayed on the Z
agar plates, and incubated overnight at 37\dg{}.

\subsection{P1 transduction of JM101}
\label{P1_transduction}

To \ml{2} of the fresh overnight culture of JM101, grown in Z
broth, 10\su{8} pfu of P1 lysate was added and the mixture was
incubated at 37\dg{} without shaking for \U{15}{min} to facilitate
phage adsorption. The unadsorbed phage particles were removed by
centrifugation at \g{4000} for \U{5}{min}. The pellet was
resuspended in \ml{5} of LB containing \mM{20} sodium citrate to
prevent further propagation of phage. The cell suspension was then
incubated at 37\dg{} for \U{30}{min} without shaking for the
phenotypic expression of the antibiotic resistance. The mixture
was then centrifuged and the pellet was resuspended in \ml{0.3} of
citrate buffer (see Section~\ref{buffers} for composition).
\mul{100} aliquots were plated on tetracycline plates,
supplemented with \mM{2.5} sodium citrate. The transduced JM101
was named as MJMRH.

\subsection{Generation of \emph{rpoS} disruption mutant}
\label{chap2_disrupt}

The \e{rpoS} disruption mutant was generated by homologous
recombination. A \U{753}{bp} internal fragment of \emph{rpoS} gene
was amplified in PCR using RPOSKOF, containing \emph{Eco}RI site,
as forward primer, and RPOSKOR, containing \emph{Hin}dIII site, as
reverse primer (Table~\ref{chap2_oligos}). The amplified product
was digested with \emph{Eco}RI and \emph{Hin}dIII and ligated to
suicide vector pME3088. The recombinant plasmid thus generated was
named as pMERPOS$'$. \mbox{S17-1} cells were transformed with
pMERPOS$'$\@. It was transferred to Lz4W in a biparental mating
between Lz4W and S17-1 carrying pME\-RPOS$'$. Because it contains
an internal fragment of \emph{rpoS}, a single recombination event
would disrupt \emph{rpoS} reading frame. For the selection of
exconjugants we took the advantage of Lz4W being naturally
ampicillin resistant. The exconjugants were selected on ampicillin
and tetracycline plates. For a schematic diagram of the whole
procedure, see Figure~\ref{chap6_rpos_disruption}.

\section{Molecular techniques}

\subsection{Agarose gel electrophoresis}

Routine checking of the DNA samples were carried out using
0.8--1\% small (\U{5}{cm} long) agarose gel, cast in either 0.5
$\times$ TAE or TBE with  ethidium bromide. Ethidium bromide was
added at final concentration of \U{0.5}{$\muup$g/ml} either
directly into molten agarose or in the tank buffer after the
electrophoresis. DNA samples were loaded in DNA loading dye
containing 30\% glycerol and 0.25\% bromophenol blue. The
electrophoresis was carried out in 0.5 $\times$ TAE or TBE. For
separation of genomic DNA digest longer gel (\U{15--20}{cm} long)
were used. For genomic DNA the electrophoresis was carried out at
constant voltage of \U{1--2}{V/cm} for \U{12--16}{h}.


\subsection{Genomic DNA purification}

Genomic DNA from bacterial cultures were purified by a modified
protocol originally presented in \citet{Towner1991}.
\unit[100]{ml} overnight grown bacterial culture was spun down and
washed with \unit[40]{ml} of TE and resuspended in \unit[3.2]{ml}
of lysis buffer containing \unit[50]{mM} Tris-Cl (pH 8.0) and
\unit[0.7]{M} sucrose. Then the following solutions were added to
the resuspended culture--- \unit[0.6]{ml} lysozyme
(\unit[20]{mg/ml}), \unit[0.6]{ml} of \unit[0.5]{M} EDTA (pH 8.0),
\unit[0.5]{ml} of 10\% SDS and \unit[5]{$\muup$l} RNase
(\unit[100]{mg/ml}). The whole mixture was incubated for
\unit[10]{min} at room temperature. Then, \unit[250]{$\muup$l} of
Proteinase K (\unit[10]{mg/ml}) was added to the mixture, and
incubated at 50\,$^\circ$C for \unit[30]{min}. The mixture was
then extracted twice with phenol:choloroform (1:1 ratio) and the
DNA precipitated by adding 1/10\su{th} volume of \unit[3]{M}
sodium acetate and 2.5 volume of ethanol. The DNA was spooled by a
glass rod and dipped several times in 70\% ethanol before
dissolving in sterile water.

The purity of the extracted DNA was then checked by standard
spectophotometric method by measuring OD\sub{260} and OD\sub{280}
of the solution. The quantity was also measured by running an
aliquot in agarose gel and comparing with the known standard.

The whole process was scaled down for extraction of DNA from
\ml{3} culture.

\subsection{Plasmid DNA purification}

Depending on the amount or the quality of plasmid DNA, various
methods were used. Small-scale (miniprep, from \ml{3} culture)
preparations were routinely done using alkaline lysis method
described below. For higher quality of plasmid DNA, commercial
kits were used. Plasmid DNA from large cultures were prepared
first by alkaline lysis followed by CsCl density gradient
centrifugation.

\subsubsection{Alkaline lysis}

Small-scale preparation of plasmid DNA was made by the alkaline
lysis method \citep{Birnboin1979} as described in
\citet{Sambrook1989} with some modifications. The first solution
for bacterial resuspension used, was TE (containing
\U{100}{$\muup$g/ml} RNase) instead of the usual (\mM{50} glucose;
\mM{25} Tris-Cl and \mM{10} EDTA).

\ml{3} bacterial culture bearing the plasmid was centrifuged.
Bacterial pellet obtained was resuspended in \mul{100} of TE
(containing \U{100}{$\muup$g/ml} RNase). \mul{200} of Solution 2
(\U{0.2}{N} NaOH, 1\% SDS) was added and the contents of the tube
were mixed by inverting the tube several times. This was followed
by the addition of \mul{150} of ice-cold Solution 3 (Made by
adding glacial acetic acid to \U{5}{M} potassium acetate till pH
becomes 4.8. For \ml{100} of solution---\ml{60} of \U{5}{M}
potassium acetate, \ml{11.5} of glacial acetic acid and \ml{28.5}
of water.) and gentle mixing. The tube was incubated on ice for 5
min and centrifuged at \U{12,000}{\emph{g}} for \U{10}{min} at
4\dg. The supernatant was extracted with an equal volume of
phenol:chloroform and the DNA was precipitated with two volumes of
absolute ethanol followed by 70\% ethanol rinse. The plasmid DNA
was then checked on a 0.8\% agarose gel and stored at -20\dg.

This DNA was suitable for routine procedures such as restriction
digestion, preparation of radiolabelled probe and manual
sequencing.

\subsubsection{Large-scale isolation of plasmid}

Large-scale preparation (from \ml{500} culture) of plasmid DNA was
carried out by CsCl density gradient centrifugation as described
in \citet{Sambrook1989} with some modifications. The major
difference was at the centrifugation step as described below.

Bacterial culture (\ml{500}) was harvested and resuspended in
\ml{8} of Solution 1 (\mM{50} glucose; \mM{25} Tris, pH 8.0 and
\mM{10} EDTA). To the suspension \mul{800} of lysozyme
(\U{10}{mg/ml}) was added. Then, \ml{16} of Solution 2 (\U{0.2}{N}
NaOH and 1\% SDS) was added to the suspension and mixed
thoroughly. The content of the tube was mixed gently by inverting
the tube. \ml{12} of Solution 3 (see Section \ref{buffers}) then
added to the tube and centrifuged at \U{12,000}{\emph g} for
\U{15}{min}. Supernatant was collected and the plasmid DNA was
precipitated by adding 0.6 volume of isopropanol. The DNA pellet
was rinsed with 70\% (\F{v}{v}) ethanol and air-dried.

For CsCl density gradient centrifugation the DNA was resuspended
in exactly \ml{4} of TE. To this solution, exactly \U{4.5}{g}
solid CsCl (Serva) was added followed by \ml{0.5} of ethidium
bromide (\U{10}{mg/ml}). The content of the tube was mixed and
centrifuged at \U{12,000}{\textit{g}} for \U{15}{min} at room
temperature. The clear solution from the top of the tube was
loaded in Beckman Quick-seal tubes, sealed and centrifuged at
\U{72,000}{rpm} for \U{5.5}{h} in a VTI80 rotor in Beckman L8-80
ultra-centrifuge.

After centrifugation, the supercoiled plasmid band was collected
by piercing the tube with hypodermic syringe. Ethidium bromide was
removed from the DNA by extracting several times with 1-butanol,
and the DNA solution was dialized against two liters of TE for
\U{24}{h}. After dialysis the DNA was precipitated with standard
ethanol precipitation method.

The DNA isolated by procedure gave the best possible quality
plasmid DNA.


\subsubsection{Purification by commercial kits}

Plasmids were prepared from \ml{3} and \ml{100} cultures by
Wizard{\scriptsize\texttrademark} \textit{Mini} and
\textit{Midi\-prep} systems from Promega Corporation, USA,
according to manufacturer's protocol. DNA prepared through these
commercial kits were mainly used for automated DNA sequencing.

\subsection{Restriction digestion}

All the restriction digestion were performed according to the
protocol supplied by the manufacturer, typically, in a reaction
volume of \mul{20--50}.

\subsection{Dephosphorylation of vector}
\label{dephos} When cut with single enzyme or when blunt ended,
the vector was dephosphorylated  using shrimp alkaline phosphatase
(SAP) before ligation. Digested vector DNA ($\sim$\U{50}{ng}) was
dissolved in \mul{7} water and to this \mul{1} dephosphorylation
buffer (10 $\times$, supplied) and \mul{1} SAP (1 unit) was added.
The mixture was incubated for \U{10}{min} at 37\dg{} for
staggered-ended vector, and \U{60}{min} at 37\dg{} for blunt-ended
vector. The SAP was inactivated for \U{15}{min} at 65\dg{}. The
dephosphorylated vector was used directly for ligation.

\subsection{Ligation}

Typically, \U{50--100}{ng} of DNA was used in each ligation
reaction in \mul{10--20} reaction volume with 0.05 Weiss units of
T4 DNA ligase. The vector to insert ratio was maintained at 1:1 or
1:3. The reaction was carried out routinely at 16\dg{} for
\U{12--16}{h}.

\subsection{Extraction of DNA from agarose gel}
\label{mat:gel-extract}

DNA fragments were purified from agarose gel by GENE
CLEAN{\scriptsize\texttrademark} (BIO 101) or
QiaQuick{\scriptsize\texttrademark} (QIAGEN) gel extraction kit
according to manufacturer's protocol.

\subsection{Making of partial genomic library}

Partial genomic library of Lz4W was made in plasmids to clone
various genes of interest. In this method, genomic DNA was
digested with several restriction enzymes and the digested DNA was
analyzed by Southern hybridization (Section \ref{mat:southern})
with a suitable probe to find out the size of the restriction
fragment that hybridizes to the probe. The enzyme that produced
clonable fragment (up to \U{6}{kb}) was chosen to make the
library. If no enzyme gave the proper size band the DNA was
digested with two enzymes to bring the size of the band down to
clonable range. The size of the fragment was noted very carefully.

Large quantity ($\sim$\mug{100}) of genomic DNA was then digested
and separated in 0.8\% TAE agarose gel in sevaral lanes. A
suitable molecular weight marker was loaded in one of the lanes.
The gel was cast without ethidium bromide and was run in constant
voltage (\U{2--3}{V/cm}) overnight. After the run, the molecular
weight marker lane was cut off from the gel and stained separately
by soaking in ethidium bromide containing tank buffer and the
marker band positions were marked on a transparent polythene
sheet. The sheet was then placed on top of the rest of the gel,
and around \U{1}{cm} thick slices were cut off from the required
region of the gel using the molecular weight as guide. DNA was
then isolated from these slices as described in Section
\ref{mat:gel-extract}. A small aliquot from each of these size
fractionated DNA samples was run in agarose gel and hybridized
with the same probe in Southern hybridization to check for the
presence of the required DNA fragment.

Approximately, \U{50}{ng} of DNA was then ligated with \U{50}{ng}
of pBluescriptII KS+ vector, cut with suitable restriction enzyme
(for vector cut with single enzyme, it was dephosphorylated as
described in Section \ref{dephos}). Around \mul{1--2} of the
ligation mix was electroporated into \bact{Ec} DH10B cells and
plated on LB plates with ampicillin. The positive clone was then
detected using colony blot, described in Section
\ref{colony-blot}.

\subsection{Cosmid genomic library}
\label{chap2:cosmid_library}

A cosmid genomic library from Lz4W was made in cosmid
pLAFR3~\citep{Staskawicz1987}. The overall protocol followed was
as described in \citet{Asubel1991}. The enzyme \emph{Pst}I was
chosen for making the library for the random distribution of its
recognition sites in Lz4W genome. The cosmid pLAFR3 was the cosmid
of choice because of the presence of a unique \emph{Pst}I site in
its MCS and its broad host-range replicon. The various steps of
making the cosmid library is described below.

\subsubsection{Standardization of partial digestion}

The partial digestion of the genomic DNA was standardized by
incubating a fixed amount of genomic DNA with various units of
\emph{Pst}I (2--0.007 units). Briefly, \mug{20} of genomic DNA in
\mul{22} volume was mixed with \mul{22} of 10$\times$ restriction
enzyme buffer and \mul{176} of water. From the mixture, \mul{20}
aliquot each, was distributed in 11 tubes. \emph{Pst}I (8 units)
was added in the first tube and the content was mixed by
pipetting. From the content, \mul{20} was transferred to the
second tube and the serial dilution was carried out to tenth tube,
with each time \mul{20} content was transferred to the next. The
first tube was discarded and the eleventh tube was kept aside as
negative control (minus enzyme). All the ten tubes were incubated
at 37\dg{} for one hour. The samples were run in 0.3\% agarose gel
with a 1\% agarose base, with a very low constant voltage (3 V/cm)
for 6 h. The concentration of the enzyme that produced the maximum
product at at around \U{20}{kbp} region was chosen for the next
step.

\subsubsection{Large-scale partial digestion of DNA}

For large-scale partial digestion about \mug{200} of genomic DNA
was taken in \mul{400} of 1$\times$ restriction enzyme buffer.
Digestion was carried out with 30 units of \emph{Pst}I at 37\dg{}
for 1 h. An aliquot of the sample was checked for digestion by
running in 0.3\% agarose gel. The same reaction was performed in
batches of four.

\subsubsection{Size fractionation of restriction fragments}

Four \ml{12} continuous gradient of NaCl (5--25\%) were prepared
in Beckman SW41 tubes using a commercial gradient maker. \mug{200}
of partially digested DNA was loaded onto each gradient and
centrifuged at \U{37,000}{rpm} for 4.5 h at 4\dg{} in Beckman SW41
rotor. Fifty fractions, each of \mul{250}, were collected from
each tube. The fractions were analyzed on 0.3\% agarose gel. The
fractions of size \U{20--30}{kb} were pulled together and
precipitated with ethanol.

\subsubsection{Preparation of cosmid DNA and ligation}

About \mug{30} of cosmid DNA (pLAFR3) was digested with
\U{40}{units} of \emph{Pst}I in a reaction volume of \mul{150}.
The digestion was carried out at 37\dg{}, overnight. The
completion of the digestion was checked by analyzing an aliquot in
0.8\% agarose gel.

The linearized cosmid was dephosphorylated by adding \mul{2}
(\U{10}{units/$\muup$l}) of calf intestinal phosphatase (CIP)
directly into the digestion tube and incubating the DNA at 37\dg{}
for \U{30}{min}. After the reaction, the digested and
dephosphorylated cosmid was purified by phenol:choloform
extraction and ethanol precipitation.

For ligation, \mug{9} of insert and \mug{3} of vector were
precipitated together by ethanol, and resuspended in \mul{10.5} of
1$\times$ ligation buffer. The ligation was carried out at
15\dg{}, overnight in \mul{15} reaction volume, in presence of 4
units of T4 DNA ligase. The efficiency of ligation was checked by
analyzing an aliquot on 0.5\% agarose gel.

\subsubsection{Packaging}

Packaging of ligated cosmid and inserts were carried out reaction
was done using Gigapack II (Stratagene) packaging extract as
specified by supplier.

\subsubsection{Titering the library}

Culture (\ml{50}) of \bact{Ec} DH10B cells were grown in LB
supplemented with \mul{500} of \U{1}{M} MgSO$_{4}$ and \mul{500}
of 20\% maltose. The cells were harvested and resuspended in
\ml{12.5} of \mM{10} MgSO$_{4}$ and stored at 4\dg{} until further
use. Just before use the cells were diluted to OD$_{600}$ of 0.5
in \mM{10} MgSO\sub{4}. Three dilutions of the packaged DNA (1:10,
1:30, 1:50) were prepared with SM buffer (see
Section~\ref{buffers} for composition). \mul{25} of each dilution
was mixed with \mul{25} of diluted DH10B cells and incubated at
room temperature for 30 min. About \mul{200} of LB broth was added
to each sample and incubated at 37\dg{} for 1 h. After the
incubation the cells were harvested by centrifugation for 1 min,
and resuspended in \mul{800} of fresh LB\@. Aliquot of \mul{200}
each, of the resuspended cells was plated on four plates. The
dilution of 1:30 was found to be suitable for amplification.

\subsubsection{Amplification and storage of the library}

The packaged phage particles (\mul{20}) were diluted to \mul{600}
in SM buffer and incubated with \mul{600} of \bact{Ec} DH10B cells
(OD\sub{600} $\sim$0.5) and incubated at room temperature for 30
min. Four volumes of LB (\ml{4.8}) was added to the cells and
incubated \U{1}{h} at 37\dg{} with shaking. The cell suspension
was distributed in four microfuge tubes and cells were pelleted
down by quick spin and resuspended into \ml{1} of LB for each
tubes. The cells were spread on twenty \U{153}{mm} LB agar plates
supplemented with tetracycline, X-gal and IPTG.

White colonies were picked up from each plates and individual
clones were inoculated into \mul{100--150} of LB containing
tetracycline in 96-well microtiter plates. The plates were
incubated overnight at 37\dg{}. Equal volume of 40\% glycerol was
added to each well and the plates were stored at $-$70\dg{}.
Around 2,200 individual clones were stored frozen by this method.

\subsubsection{Screening of the library}

A custom-made inoculation devise, with 96 projections was used to
make replica of the 96-well microtiter plates on
Hybond\texttrademark N\su{$+$} nylon membrane. Each membrane were
treated and probed as discussed in Section~\ref{colony-blot}.


\subsection{Transformation}

Depending on the transformation efficiency required, various
transformation protocols were used for \textit{E. coli}.
\textit{P. syringae} cells were transformed only by
electroporation.

\subsubsection{Calcium chloride method}

For routine plasmid transformations, where efficiency of
transformation were not important, the ``classical''
\citet{Mandel1970} calcium-chloride method were used. The
following protocol was a modification of the protocol described in
\citet{Brown1991}.

An overnight grown culture of the \bact{Ec} was diluted 100 times
in fresh LB medium (\unit[100]{$\muup$l} cells inoculated in
\unit[10]{ml}) and subcultered till OD\sub{600} reached 0.4--0.5.
The culture was chilled on ice for \unit[15]{min}. All the steps
thereafter were carried out at 4\,$^\circ$C. The cells were
harvested and resuspended in \nicefrac{1}{2} volume of the
original culture of sterile ice-cold \unit[100]{mM} CaCl\sub{2}.
The cells were again harvested and resuspended in \nicefrac{1}{10}
volume of ice-cold \unit[100]{mM} CaCl\sub{2}\@. This suspension
was incubated on ice for at least \unit[1]{h}\@. To
\unit[100]{$\muup$l} of this suspension, DNA (\unit[10--100]{ng}
of DNA in less than \unit[10]{$\muup$l} volume) was added. The
mixture was incubated on ice for another \unit[30]{min} and then
transferred to a 42\,$^\circ$C water bath for exactly
\unit[90]{s}. Immediately, \unit[0.9]{ml} of LB medium was added
to the tube and the tube was incubated at 37\,$^\circ$C for
\unit[30--45]{min} for phenotypic expression of the antibiotic
marker before plating them on selective media at various
dilutions.

\subsubsection{Rubidium chloride method} \label{rubidium}

Most of the routine cloning experiments were carried out with
competent cells made according to the method described by
\citet{Hanahan1985} with a few modifications. In this protocol a
single colony of \bact{Ec} cells was inoculated into \ml{3} of LB
medium and incubated at 37\dg{} overnight. Aliquot of \mul{500} of
this overnight culture was inoculated into \ml{50} of SOB medium
and incubated at 37\dg{} at \U{200}{rpm} till the OD\sub{600}
reached $\sim$0.5. The culture was then chilled on ice for
\U{15}{min} and the cells were harvested by centrifugation at
\U{4000}{rpm} for \U{15}{min} at 4\dg{}. The supernatant was
completely drained and the pellet was gently resuspended in
\ml{20} (0.4 volume) of ice-cold RF1 buffer (see Section
\ref{buffers}) and incubated on ice for \U{20}{min}. The cells
were again harvested and resuspended in 2 ml (0.04 volume) of the
ice-cold RF2 buffer (Section \ref{buffers}) and incubated on ice
for \U{15}{min}. The cells were then distributed into prechilled
microfuge tubes in \mul{40} aliquots; flash frozen in liquid
nitrogen and stored at \mbox{$-$70\dg}\@. Whenever required, one
aliquot of cells was thawed on ice and mixed with DNA
(\U{50--100}{ng}) and incubated for \U{30}{min} on ice. After
incubation the tube was transferred to 42\dg{} water bath for
\U{90}{s} and \ml{1} of LB or SOB or SOC added to the tube and
incubated at 37\dg{} for \U{30--60}{min} and plated on selection
plate.

The competent cells prepared by this method generally yielded a
transformation efficiency of 5 $\times$ 10\su{6} to 1 $\times$
10\su{7} colonies/$\muup$g of pUC19 DNA, and were stable up to
three months at \mbox{$-$70\dg{}}.

\subsubsection{Ultracompetent cells}

Ultracompetent cells were used for cloning experiments requiring
high efficiency transformation. The cells were prepared according
to the method described in \citet{Inoue1990}. A single colony of
\bact{Ec} recipient cell was inoculated into \ml{3} of LB and
incubated at 37\dg{} overnight. Aliquot of \ml{2.5} of this
overnight culture was inoculated into \ml{250} of SOB medium and
incubated at 18\dg{} at \U{200}{rpm}, till the OD\sub{600} reached
0.6--0.7. The culture was chilled on ice and the cells were
harvested by centrifugation at \U{4000}{rpm} for \U{15}{min} at
4\dg{}\@. The cells were then resuspended in \ml{80} of ice-cold
transformation buffer (TB, see Section \ref{buffers}) and
incubated on ice for \U{10}{min}. The cells were harvested and
resuspended in \ml{20} of TB containing 7\% DMSO. The cells were
then distributed in \mul{100} aliquots; flash frozen in liquid
nitrogen and stored at -70\dg{}. The transformation were done by
heat-shock as in case of rubidium chloride method discussed in
Section \ref{rubidium}.

Ultracompetent cells prepared by this method yielded a
transformation efficiency of 1 $\times$ 10\su{8} colonies/$\muup$g
of pUC19 DNA and were stable for six months at -70\dg.

\subsubsection{Electroporation}

For making plasmid genomic library  or when very high efficiency
of transformation was required the \textit{E. coli} cells were
transformed by electroporation.

The recipient cells were made \textit{electrocompetent} by the
following method. Cells were grown in large volume (one liter)
till the OD\sub{600} reaches 0.5--0.6. Culture was then cooled and
washed three to five times with sterile ice-cold water (the total
washing should be atleast the same volume of the original
culture). The last washing was carried out in sterile ice-cold
10\% glycerol. The cells were then resuspended in \mul{500} of
sterile ice-cold 10\% glycerol, snap frozen in \mul{40} aliquots
in liquid nitrogen, and stored in -70\dg. The electrocompetent
cells thus prepared were transformed by electroporation using
Bio-Rad Gene Pulser{\scriptsize\texttrademark} II, according to
manufacturer's protocol.

\subsection{Preparation of radio-labelled probes}

Oligonucleotides were radio-labelled at the 5$'$-end with
$\gammaup$-[\su{32}P]dATP using T4 polynucleotide kinase (PNK)\@.
About 10 picomoles of primer was incubated at 37\dg{} for 30 min
with \U{40}{$\muup$Ci} of $\gammaup$-[\su{32}P]dATP and 5 units of
PNK in \mul{50} reaction volume.

Longer double-stranded DNAs were radio-labelled by random-priming
with $\alphaup$-[\su{32}P]dATP using Klenow enzyme and random
hexamer oligonucleotides by a commercial random-priming kit (BARC,
India), according to manufacturer's protocol. About \U{50}{ng} of
DNA was labelled in each reaction.

After the reaction, the probe was purified from the unincorporated
nucleotides by Sephadex G25 or G50 spin-column chromatography.
Sephadex G25 was used for oligonucletide probes and G50 was used
for probes longer that 150 bp. Briefly, a \ml{1} disposable
syringe was plugged with siliconized glass-wool (Supelco) and
filled with Sephadex G25/50 slurry previously equilibrated with TE
(pH 8.0). The column was packed by centrifugation at
\U{2,500}{rpm} for \U{5}{min} in a Sorvall HB-4 swing-out rotor.
Radio-labelled probe (\mul{100}) was then loaded on the column and
the probe was purified by centrifugation, again at \U{3,000}{rpm}
for \U{5}{min}.

\subsection{Southern hybridization}
\label{mat:southern}

Southern hybridization was performed as described in
\citet{Parry1995}. Briefly, DNA samples were electrophoresed in
appropriate agarose gels. After electrophoresis, the DNA was
transferred onto positively charged nylon membrane by capillary
transfer overnight in \unit[0.4]{N} NaOH, a method originally
described in \citet{Reed1985}. After transfer, the membrane was
rinsed with 2 $\times$ SSC air-dried and stored at room
temperature until further use.

Hybridization was carried out in buffer described in
Section~\ref{buffers} \citep{Church1984} at 65\,$^\circ$C,
overnight for double-stranded probes and 55\dg{} overnight for
oligonucleotide probes in sealed polythene bag. After the
hybridization, the membrane was washed to remove the unbound
probe. In the case of oligonucleotide probe, the membrane was
washed with 6 $\times$ SSC thrice at room temperature with each
wash lasting for 20 min, and in the case of longer probes the
membrane was washed with 0.5--0.1 $\times$ SSC with 0.1\% SDS at
68\dg{} for appropriate times and then exposed to phosphorimager
(Fuji) screen or X-ray film.

\subsection{Colony blot}
\label{colony-blot}

Genomic library of Lz4W in plasmid or cosmid, and transformed
cells were screened for positives using colony blotting. In the
case of genomic library the colonies were either grown or spread
on Hybond\texttrademark-N\su{$+$} membrane placed on agar surface.
In some cases already grown colonies were lifted by placing the
membrane on top of the colonies on agar surface. After colonies
were transferred on the membrane, it was placed, colony side up,
on a pad of Whatman 3M filter paper soaked in denaturing solution
(\U{1.5}{M} NaCl; \U{0.5}{M} NaOH) for \U{7}{min}. The membrane
was then placed on another pad of filter paper soaked in
neutralizing solution (\U{1.5}{M} NaCl; \U{0.5}{M} Tris-HCl pH
7.2; \U{0.001}{M} EDTA) for 3 min. The last step was repeated
again by changing the filter paper. The membrane was then washed
vigorously in 2 $\times$ SSC and wiped gently with a cotton ball
soaked in 2 $\times$ SSC to remove cell debris. The membrane was
then transferred on a dry filter paper and air-dried. The DNA was
fixed on membrane using alkali fixation procedure~\citep{Reed1985}
by placing it on the surface of a filter paper soaked in
\U{0.4}{M} NaOH for 20 min. The membrane was then rinsed with 2
$\times$ SSC for \U{1}{min}, air-dried and stored at room
temperatures until further use. The hybridization was carried out
using conditions described in Section~\ref{mat:southern}.



\subsection{Polymerase Chain Reaction (PCR)}

For PCR using genomic DNA as a template about \U{250}{ng} DNA was
used as template and for plasmid DNA \U{10--20}{ng} DNA was used
as template in \mul{50 or 100} reaction volume. Reactions were
performed with \muM{200} of dNTPs, \U{1}{picomoles/$\muup$l} each
of forward and reverse primers and \U{1.5}{units} of \emph{Taq}
DNA polymerase.

For genomic PCR, the cycle conditions were as follows: initial
\U{3}{min} denaturation at 94\dg{}. Three cycles of denaturation
at 94\dg{} for \U{1}{min}, annealing at 37\dg{} for \U{1}{min},
and extension at 72\dg{} for \U{2}{min}. Followed by 25 cycles of
denaturation at 94\dg{} for \U{1}{min}, annealing at 50\dg{} for
\U{1}{min}, and extension at 72\dg{} for \U{2}{min}. The reaction
was terminated with \U{5}{min} extension at 72\dg.

For plasmid DNA template, the cycle conditions were as follows:
initial \U{5}{min} denaturation at 95\dg. Then 30 cycles with
denaturation at 95\dg{} for \U{1}{min}, annealing at 50--55\dg{}
for \U{1}{min} and extension at 72\dg{} for \U{1--2}{min}. The
reaction was terminated with extension at 72\dg{} for \U{5}{min}.

The PCR reactions were performed in either Perkin Elmer or M J
Research's DNA Engine{\scriptsize{\texttrademark}} thermal cycler.

\subsection{DNA sequencing}

The DNA sequencing reaction was carried out on double stranded
plasmid template using dye terminator chemistry using Big Dye
Terminator sequencing kit (Applied Biosystems, USA) and the
sequences determined using either ABI Prism 3700 or ABI prism 377
automated DNA sequencer (Applied Biosystems, USA).

\subsection{SDS-PAGE}
\label{mat:sds} Whole cell proteins were separated on 10--12\%
SDS-PAGE according to the \emph{classical} protocol devised by
\citet{Laemmli1970} as described in \citet{Sambrook1989}.

For separating whole cell proteins, cells were harvested from
various growth phases and resuspended at calculated OD of 20 in 1
$\times$ sample buffer (see Section~\ref{buffers} for
composition). This cell suspension was boiled for \U{5}{min} and
debris was removed by centrifuging at \g{12,000} for \U{15}{min}.
Clean supernatant in a volume of \mul{25--50} was loaded in each
lane.

For analysis of cell extract, measured volume of crude extracts
containing equal amounts of protein were taken in \mul{20} of
water. Five microliters of 5 $\times$ sample buffer was added to
each tube. The samples were boiled for \U{5}{min} and equal amount
loaded in each lane.

Gels were run at constant current of \U{20}{mA} for stacking gel,
and \U{40}{mA} for resolving gel. After the run, if the gel was
for western blotting, it was processed as described in Section
\ref{mat:immunoblot}; otherwise, the gel was stained in Coomassie
Brilliant Blue (\U{0.25}{g} in 30\% methanol and 10\% acetic acid)
for minimum \U{1}{h} and destained in 30\% methanol and 10\%
acetic acid mixture.

\subsection{Immunoblotting with anti-RpoS antibody}
\label{mat:immunoblot} RpoS protein level was detected using a
mouse anti-RpoS polyclonal serum generated against \emph{P.
putida} RpoS~\citep[kind gift from M. Kivisaar]{Ojangu2000}.
Polyclonal anti-sera generated against \bact{Ec} RpoS (kind gift
of R. Hengge-Aronis) did not give consistent result. For
comparative quantitation of RpoS, protein samples were prepared by
either sonication in phosphate buffer, as described in
Section~\ref{crude} or by direct cell lysis in SDS-PAGE sample
buffer. The equalization of the protein amount in the samples were
done by either measured protein concentration or by resuspending
the cell pellet to an estimated final OD of 20 in 1$\times$
SDS-PAGE sample buffer. Typically \mul{20--40} of samples were
loaded in each lane.

After separation on SDS-PAGE (see Section~\ref{mat:sds}), the
samples were transferred on to Hybond{\tiny\texttrademark} PVDF
(polyvinyledene difluride, Amersham) or nitrocellulose
(Hybond{\tiny\texttrademark} C, Amersham) using BIO-RAD
Trans-Blot{\tiny\texttrademark} semi-dry transfer apparatus, in
transfer buffer, described in Section~\ref{buffers}. The transfer
was carried out for \unit[1.5]{h} with current set at
\unit[0.8]{mA/cm$^{2}$}. After transfer, the efficieny was checked
by staining the blot in 0.5\% Ponceau S (Sigma) in 1\% acetic
acid. The excess stain was removed by rinsing the blot several
times with water and the positions of molecular weight marker
proteins were marked on the blot. If required, the blot was then
scanned using a flat-bed scanner for reference.

Next, the blot was blocked in 5\% not-fat milk in TBS-T (see
Section \ref{buffers}) for \unit[2--12]{h}. The blot was then
washed three times with TBS-T and incubated with primary antibody
(1:250 dilution, in TBS-T with 1\% BSA) for \unit[1.5--2]{h}.
After incubation, the blot was washed thrice with TBS-T, each wash
for\unit[15]{min}. The blot was then incubated in anti-mouse IgG
secondary antibody (in TBS-T with 1\% BSA, with 1:10,000 dilution)
for \unit[1]{h}. The blot was subsequently washed three times with
TBS-T to get rid of excess antibody. After rinsing with water, the
blot was incubated in \unit[10]{ml} of AP buffer (see Section
\ref{buffers}) in presence of \unit[66]{$\muup$l} of NBT (5\%
solution in DMF) and \unit[33]{$\muup$l} of BCIP (disodium salt;
5\% aqueous solution) till the signal appeared.\ding{45}
